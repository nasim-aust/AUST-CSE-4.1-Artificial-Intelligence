\documentclass[11pt]{article}
%Gummi|065|=)
\title{\textbf{CSE 4108 - Artificial Intelligence Sessional}}
\date{Spring 2018}
\author{Python Resources}

\usepackage{hyperref}
\usepackage{url}
\usepackage{array}
\usepackage{makecell}

\renewcommand\theadalign{bc}
\renewcommand\theadfont{\bfseries}
\renewcommand\theadgape{\Gape[4pt]}
\renewcommand\cellgape{\Gape[4pt]}

\begin{document}
\maketitle


\section{Python books and softwares}

\subsection{Distributions for Python}
\begin{description}
\item [Anaconda] Download and install \textbf{Anaconda}, a scientific Python distribution, from \url{https://www.anaconda.com/download/}. Remember to tick the box that adds path variable (at the end of the installation)
\item [IDE] Using \textbf{Spyder} or \textbf{Jupyter Notebook} as IDE for python is encouraged. \textbf{Pycharm} can also be an alternative (however, it is slower).
\item [Python Basic Installation] If you install Python 3.5+ from Python site, you will get an IDLE, along with tutorials and documentation PDF.
\end{description}


\subsection{Books for Python}
\begin{description}
\item [Learning Basic Python] The documentation of Python (\url{https://docs.python.org/3/}) is almost self-sufficient for learning the basic data structures, control statements and OOP of Python.

\item [Learning not-so-basic-not-so-intermediate Python] For learning python:  \textit{Automate boring stuffs with Python} \url{https://goo.gl/rvbw5Z}. This book is about practical hands-on approach to Python.

\item [Learning numpy] A great library for python is numpy. Learn it from here: \url{http://cs231n.github.io/python-numpy-tutorial/}

\item [Learning from Jupyter Notebooks]
 A good collection of notebooks (look up the section called library. Also, you can run them using Jupyter Notebooks) \url{https://github.com/jdwittenauer/ipython-notebooks}


\end{description}





\end{document}
